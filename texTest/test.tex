\documentclass[12pt]{article}
\usepackage{fontspec}
\usepackage{xeCJK}
\usepackage{geometry}
\usepackage{fancyhdr}
\usepackage{amsmath, amssymb}
\usepackage{graphicx}
\usepackage{booktabs}
\usepackage{hyperref}
\usepackage{listings}
\usepackage{xcolor}
\usepackage{longtable} % ✅ 加上这个,避免 longtable 报错

% 页面设置
\geometry{a4paper, margin=2.5cm}
\setmainfont{Times New Roman}
\setCJKmainfont{SimSun}        % 宋体
\setCJKmonofont{SimHei}        % 黑体
\setCJKsansfont{FangSong}      % 仿宋体
\linespread{1.5}

% 页眉页脚
\pagestyle{fancy}
\fancyhf{}
\fancyhead[L]{中文测试文档}
\fancyhead[R]{\thepage}

% 超链接颜色
\hypersetup{
    colorlinks=true,
    linkcolor=blue,
    urlcolor=cyan
}

% 代码样式
\lstset{
  backgroundcolor=\color{gray!10},
  basicstyle=\ttfamily\small,
  keywordstyle=\color{blue},
  commentstyle=\color{green!50!black},
  stringstyle=\color{red},
  numbers=left,
  numberstyle=\tiny,
  stepnumber=1,
  frame=single,
  breaklines=true
}

% 正文开始
\title{\Huge 中文 LaTeX 测试文档\par \vspace{1em} \Large(适用于 XeLaTeX 编译)}
\author{\large 作者:张三\\ 学号:123456\\ 专业:交通运输}
\date{\today}

\begin{document}

\maketitle
\cleardoublepage

\tableofcontents
\cleardoublepage

\section{引言}
这是一个用 \LaTeX{} 撰写的中文文档示例。本文旨在展示中文的支持能力,并测试不同的排版元素。

\section{段落与注释}
你可以使用多种方式输入中文段落。例如:

中国是一个有着悠久历史的国家。\footnote{这是一条脚注说明。}她的文化源远流长,科技发展日新月异。

我们可以通过 \LaTeX{} 精确控制排版结构,实现优美、规范的学术文档。

\section{数学公式}
行内公式示例:著名的爱因斯坦质能方程 $E = mc^2$。

独立公式示例:
\[
f(x) = \int_{-\infty}^{\infty} e^{-x^2} dx
\]

\section{列表展示}
\subsection{无序列表}
\begin{itemize}
    \item 项目一:数据采集
    \item 项目二:数据预处理
    \item 项目三:建模分析
\end{itemize}

\subsection{有序列表}
\begin{enumerate}
    \item 引言
    \item 方法
    \item 结果
    \item 结论
\end{enumerate}

\section{表格示例}
\begin{table}[h]
    \centering
    \caption{实验数据统计}
    \begin{tabular}{lccc}
        \toprule
        类别 & 平均值 & 最大值 & 最小值 \\
        \midrule
        A类 & 23.5 & 45 & 10 \\
        B类 & 19.8 & 40 & 8 \\
        C类 & 30.1 & 55 & 15 \\
        \bottomrule
    \end{tabular}
\end{table}

\section{图片插入}
\begin{figure}[h]
    \centering
    \includegraphics[width=0.5\textwidth]{example-image}
    \caption{这是一个示例图片}
\end{figure}

\section{文件解析}
\begin{longtable}{@{}ll@{}}
    \toprule
    \textbf{文件名} & \textbf{说明} \\
    \midrule
    \texttt{test.tex} & LaTeX 源文件,包含文档内容。 \\
    \texttt{test.pdf} & 编译后生成的 PDF 文件,即最终文档。 \\
    \texttt{test.aux} & 辅助文件,存储交叉引用信息。 \\
    \texttt{test.log} & 编译日志,记录编译过程中的信息、警告与错误。 \\
    \texttt{test.out} & 输出文件,可能包含章节或标题等信息(依赖文档类)。 \\
    \texttt{test.synctex.gz} & 用于源代码与 PDF 同步跳转的文件。 \\
    \texttt{test.toc} & 目录文件,用于生成文档的目录结构。 \\
    \bottomrule
\end{longtable}

\section{代码高亮}
下面是一个 Python 示例代码:
\begin{lstlisting}[language=Python]
def factorial(n):
    if n == 0:
        return 1
    else:
        return n * factorial(n-1)
\end{lstlisting}

\section{引用与链接}
我们可以引用其他部分,比如参见第 \ref{sec:conclusion} 节。

也可以添加超链接,例如访问 \href{https://www.latex-project.org}{LaTeX 官网}。

\section{结论与未来工作} \label{sec:conclusion}
本文展示了 LaTeX 在中文环境下的使用方法,涵盖了排版、公式、图表、代码等常见需求,适用于学术写作与报告。

未来工作可包括:
\begin{itemize}
    \item 添加参考文献
    \item 整合 BibTeX 管理
    \item 使用 Beamer 制作幻灯片
\end{itemize}

\end{document}
