\documentclass[12pt]{ctexart} 
% 使用中文文档类 ctexart,适合撰写中文文章,12pt 表示设置正文主字体大小为12磅

% ========== 常用宏包 ==========
\usepackage{amsmath, amssymb}   % 数学公式与符号支持
\usepackage{graphicx}           % 插图支持,如插入 PNG、PDF 等图片
\usepackage{geometry}           % 设置页面尺寸与边距
\usepackage{tikz}               % 画图宏包,可绘制坐标图、函数图等
\usepackage{pgfplots}           % 高级绘图支持(关键添加)
\usepackage{caption}            % 图表标题样式美化
\usepackage{fancyhdr}

% 页眉页脚
\pagestyle{fancy}
\fancyhf{}
\fancyhead[L]{110gituser}
\fancyhead[R]{\thepage}

% ========== PGFPlots 配置 ==========
\pgfplotsset{compat=1.18}       % 设置兼容版本(避免警告)
\usetikzlibrary{arrows.meta}    % 箭头样式库(可选)

% ========== 样式设置 ==========
\CTEXsetup[format={\raggedright\bfseries\zihao{4}}]{section} 
% 设置 section 标题格式:
% - \raggedright 左对齐
% - \bfseries 加粗
% - \zihao{4} 使用中文四号字体(约12pt)

\geometry{a4paper, margin=1in}
\title{LM0:使用拉格朗日乘数法求解约束优化问题}
\author{110gituser}
\date{}

\begin{document}

\maketitle

\section{问题描述}

本问题的目标是在单位圆约束下,最小化一个二维函数:
\[
\min_{x_1, x_2} \quad f(x_1, x_2) = x_1^2 + 2x_2
\]
约束条件为:
\[
g(x_1, x_2) = x_1^2 + x_2^2 - 1 = 0
\]
该约束表示解必须在单位圆上。

\section{拉格朗日乘数法原理}

拉格朗日乘数法是一种用于求解带等式约束的最优化问题的方法。其基本思想是:若函数 \( f(x_1, x_2) \) 在约束 \( g(x_1, x_2) = 0 \) 下有极值,则在极值点处,目标函数的梯度 \( \nabla f \) 与约束函数的梯度 \( \nabla g \) 必须共线,即存在一个实数 \( \lambda \),使得:
\[
\nabla f(x_1, x_2) = \lambda \nabla g(x_1, x_2)
\]

我们引入拉格朗日函数:
\[
\mathcal{L}(x_1, x_2, \lambda) = f(x_1, x_2) - \lambda g(x_1, x_2)
\]

对其求偏导并令其为零,得到以下方程组:
\[
\begin{cases}
\frac{\partial \mathcal{L}}{\partial x_1} = 2x_1 - 2\lambda x_1 = 0 \\
\frac{\partial \mathcal{L}}{\partial x_2} = 2 - 2\lambda x_2 = 0 \\
\frac{\partial \mathcal{L}}{\partial \lambda} = - (x_1^2 + x_2^2 - 1) = 0
\end{cases}
\]

求解该方程组可以得到所有可能的驻点。再将这些点代入目标函数 \( f(x_1, x_2) \) 中,比较其函数值即可确定极小值点。

\section{可视化分析}

\begin{figure}[htbp]
  \centering
  \begin{tikzpicture}
    \begin{axis}[
        width=0.8\linewidth,
        height=0.6\linewidth,
        xlabel={$x_1$},
        ylabel={$x_2$},
        zlabel={$f(x_1,x_2)$},
        view={60}{30},
        colormap/viridis,       % 添加颜色映射
        title={目标函数与约束的可视化},
    ]
      % 目标函数曲面
      \addplot3[
        surf,
        domain=-1.5:1.5,
        y domain=-1.5:1.5,
        samples=30,
        opacity=0.7,
      ] {x^2 + 2*y};
      
      % 约束条件(单位圆在曲面上的投影)
      \addplot3[
        red,
        thick,
        samples=100,
        domain=0:2*pi,
      ] ({cos(deg(x))}, {sin(deg(x))}, {cos(deg(x))^2 + 2*sin(deg(x))});
      
      % 最优解标记
      \addplot3[
        only marks,
        mark=*,
        mark size=3pt,
        black,
      ] coordinates {(0, -1, -2)};
      
      \node at (axis cs:0.5,-1.2,-2.5) {$(0,-1)$};
    \end{axis}
  \end{tikzpicture}
  \caption{目标函数 $f(x_1,x_2)=x_1^2+2x_2$ 与单位圆约束的 3D 可视化。红色曲线为约束,黑点为最优解。}
  \label{fig:optimization}
\end{figure}

\section{结论}

该方法清晰地展示了如何在非线性约束下寻找目标函数的极值。通过引入拉格朗日乘子,我们将约束优化问题转化为无约束的方程组求解问题,便于理论分析与数值实现。

\end{document}