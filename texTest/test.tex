\documentclass[12pt]{article}
\usepackage{fontspec}
\usepackage{xeCJK}
\usepackage{geometry}
\usepackage{fancyhdr}
\usepackage{amsmath, amssymb}
\usepackage{graphicx}
\usepackage{booktabs}
\usepackage{hyperref}
\usepackage{listings}
\usepackage{xcolor}
\usepackage{longtable} % 加上这个,避免 longtable 报错

% 页面设置
\geometry{a4paper, margin=2.5cm}
\setmainfont{Times New Roman}
\setCJKmainfont{SimSun}        % 宋体
\setCJKmonofont{SimHei}        % 黑体
\setCJKsansfont{FangSong}      % 仿宋体
\linespread{1.5}

% 页眉页脚
\pagestyle{fancy}
\fancyhf{}
\fancyhead[L]{中文测试文档}
\fancyhead[R]{\thepage}

% 超链接颜色
\hypersetup{
    colorlinks=true,
    linkcolor=blue,
    urlcolor=cyan
}

% 代码样式
\lstset{
  backgroundcolor=\color{gray!10},
  basicstyle=\ttfamily\small,
  keywordstyle=\color{blue},
  commentstyle=\color{green!50!black},
  stringstyle=\color{red},
  numbers=left,
  numberstyle=\tiny,
  stepnumber=1,
  frame=single,
  breaklines=true
}

% 正文开始
\title{\Huge 中文 LaTeX 测试文档\par \vspace{1em} \Large(适用于 XeLaTeX 编译)}
\author{\large 作者:110gituser\\}
\date{\today}
\begin{document}
\maketitle
\cleardoublepage
\tableofcontents
\cleardoublepage

\section{LM0:使用拉格朗日乘数法求解约束优化问题}

本问题的目标是在单位圆约束下,最小化一个二维函数:
\[
\min_{x_1, x_2} \quad f(x_1, x_2) = x_1^2 + 2x_2
\]
约束条件为:
\[
g(x_1, x_2) = x_1^2 + x_2^2 - 1 = 0
\]
该约束表示解必须在单位圆上。

拉格朗日乘数法是一种用于求解带等式约束的最优化问题的方法。其基本思想是:若函数 \( f(x_1, x_2) \) 在约束 \( g(x_1, x_2) = 0 \) 下有极值,则在极值点处,目标函数的梯度 \( \nabla f \) 与约束函数的梯度 \( \nabla g \) 必须共线,即存在一个实数 \( \lambda \),使得:
\[
\nabla f(x_1, x_2) = \lambda \nabla g(x_1, x_2)
\]

我们引入拉格朗日函数:
\[
\mathcal{L}(x_1, x_2, \lambda) = f(x_1, x_2) - \lambda g(x_1, x_2)
\]

对其求偏导并令其为零,得到以下方程组:
\[
\begin{cases}
\frac{\partial \mathcal{L}}{\partial x_1} = 2x_1 - 2\lambda x_1 = 0 \\
\frac{\partial \mathcal{L}}{\partial x_2} = 2 - 2\lambda x_2 = 0 \\
\frac{\partial \mathcal{L}}{\partial \lambda} = - (x_1^2 + x_2^2 - 1) = 0
\end{cases}
\]

求解该方程组可以得到所有可能的驻点。再将这些点代入目标函数 \( f(x_1, x_2) \) 中,比较其函数值即可确定极小值点。


该方法清晰地展示了如何在非线性约束下寻找目标函数的极值。通过引入拉格朗日乘子,我们将约束优化问题转化为无约束的方程组求解问题,便于理论分析与数值实现。

\section{引言}
这是一个用 \LaTeX{} 撰写的中文文档示例。本文旨在展示中文的支持能力,并测试不同的排版元素。
\section{段落与注释}
你可以使用多种方式输入中文段落。例如:
中国是一个有着悠久历史的国家。她的文化源远流长,科技发展日新月异。
我们可以通过 \LaTeX{} 精确控制排版结构,实现优美、规范的学术文档。
\section{数学公式}
行内公式示例:著名的爱因斯坦质能方程 $E = mc^2$。
独立公式示例:
\[
f(x) = \int_{-\infty}^{\infty} e^{-x^2} dx
\]
\section{列表展示}
\subsection{无序列表}
\begin{itemize}
    \item 项目一:数据采集
    \item 项目二:数据预处理
    \item 项目三:建模分析
\end{itemize}
\subsection{有序列表}
\begin{enumerate}
    \item 引言
    \item 方法
    \item 结果
    \item 结论
\end{enumerate}

\section{图片插入}
\begin{figure}[h]
    \centering
    \includegraphics[width=0.5\textwidth]{example-image}
    \caption{这是一个示例图片}
\end{figure}
\section{文件解析}
\begin{longtable}{@{}ll@{}}
    \toprule
    \textbf{文件名} & \textbf{说明} \\
    \midrule
    \texttt{test.tex} & LaTeX 源文件,包含文档内容。 \\
    \texttt{test.pdf} & 编译后生成的 PDF 文件,即最终文档。 \\
    \texttt{test.aux} & 辅助文件,存储交叉引用信息。 \\
    \texttt{test.log} & 编译日志,记录编译过程中的信息、警告与错误。 \\
    \texttt{test.out} & 输出文件,可能包含章节或标题等信息(依赖文档类)。 \\
    \texttt{test.synctex.gz} & 用于源代码与 PDF 同步跳转的文件。 \\
    \texttt{test.toc} & 目录文件,用于生成文档的目录结构。 \\
    \bottomrule
\end{longtable}
\section{代码高亮}
下面是一个 Python 示例代码:
\begin{lstlisting}[language=Python]
def factorial(n):
    if n == 0:
        return 1
    else:
        return n * factorial(n-1)
\end{lstlisting}
\section{引用与链接}
我们可以引用其他部分,比如参见第 \ref{sec:conclusion} 节。
也可以添加超链接,例如访问 \href{https://www.latex-project.org}{LaTeX 官网}。

\end{document}